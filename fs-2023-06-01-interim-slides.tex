% presentation | handout
\documentclass[aspectratio=169,usepdftitle=true,handout,10pt]{beamer}

\usepackage[T1]{fontenc}
\usepackage[utf8]{inputenc}
\usepackage[english]{babel}
\usepackage{microtype}

\usepackage{etoolbox,tabularx,array}
\usepackage{fp}
\makeatletter
\appto\input@path{{lib/color-palettes/}{lib/sopra-collection/sopra-listings}{lib/code-animation/}}

\usepackage{lib/color-palettes/color-palettes}
\colorlet{paletteA}{paletteA!80!black}
\colorlet{paletteB}{paletteB!80!black}
\colorlet{paletteC}{paletteC!80!black}
\colorlet{paletteD}{paletteD!80!black}
\usepackage[cpalette,encoding,defaultfont,fakeminted]{lib/sopra-collection/sopra-listings/sopra-listings}

\solLoadLanguage{R}
\solsetmintedstyle{plain}
\usepackage{lib/code-animation/code-animation}
\SetCodeAnimationFadeOutMixin{!10!btdm@background}
\CodeAnimationsWithFadeOut

\usepackage[lightmode]{lib/beamer-themes/digital-minimal/beamerthemedigital-minimal}
\usepackage[backend=bibtex8,style=numeric]{biblatex}
\usepackage{csquotes}
\usepackage{siunitx}
\usepackage{makecell}
\let\say\enquote
\addbibresource{references.bib}
\outro{\frametitle{Open Questions} \begin{itemize}
   \itemsep10pt
   \item<2-> What other features are of interest? (if any)
   \item<2-> Which domain should we use for the quads?
   \item<2-> What parts of the dataflow-information are important?
\end{itemize}}

\usetikzlibrary{arrows.meta,backgrounds,shapes.symbols,decorations.pathreplacing,shapes.geometric}
\usepackage{forest}

\makeatletter
\usepackage[outline]{contour}
\PassOptionsToPackage{normalem}{ulem}
\usepackage{ulem}
% we do it manually to "fix" it, we do not need breaking
\DeclareRobustCommand*\fancyul[2][\fancyulbackground]{%
   \def\fancyul@background{#1}%
   \def\ULdepth{1pt}%
   \def\ULthickness{.1ex}%
   \contourlength{.35pt}%
   {\color{\fancyulcolor}\uline{\phantom{#2}}}\llap{\contour\fancyul@background{#2}}%
}
\makeatletter
\def\fancyulbackground{btdm@background}%
\colorlet{cv@ul@line@color}{btdm@primary!50!btdm@background}
\hypersetup{colorlinks=false,pdfborder=0 0 0}% just make sure :D
\def\fancyulcolor{cv@ul@line@color}%
% we make the auto detection of the depth bigger
\def\update@ul{\setbox\z@\hbox{{(j}}\let\ULdepth\p@\edef\ULthickness{\the\dimexpr.235\dp\z@}}
% define a link macro that can be used with href and fancyul
\DeclareRobustCommand*\link{\hyper@normalise\link@do}
\def\link@do#1#2{\leavevmode\hskip-2\p@\href{#1}{\update@ul\hskip2\p@\fancyul{#2}}}% thinspace for some nicer padding
\DeclareRobustCommand*\hlink{\hyper@normalise\hlink@do}
\def\hlink@do#1#2{\leavevmode\hskip-2\p@\hyperlink{#1}{\update@ul\hskip2\p@\fancyul{#2}}}%

\def\cv@link{\link}% allow for later redefinitions


% too short of a presentation
\sectionbannerfalse

\title{Interesting Patterns in R Code}
\setauthormailsubject{CodeInspector\%20|\%20Interim\%201}
\subtitle{CodeInspector --- Monthly-Meeting}
\setfootmarker{CodeInspector~\faAngleRight\ Monthly}

\date{\today}

\author[F.~Sihler]{Florian Sihler}
\email{florian.sihler@uni-ulm.de}

\errorcontextlines9999
% twoheadleftarrow
% TODO: \raisebox{.025ex}{\guilsinglleft}\kern-.65ex---
\def\LeftArrow{\text{\BeginAccSupp{method=escape,ActualText={<-}}\(\leftarrow\)\EndAccSupp{}}}
\def\RightArrow{\text{\BeginAccSupp{method=escape,ActualText={<-}}\(\rightarrow\)\EndAccSupp{}}}
\def\DoubleLeftArrow{\text{\BeginAccSupp{method=escape,ActualText={<<-}}\(\twoheadleftarrow\)\EndAccSupp{}}}
\def\DoubleRightArrow{\text{\BeginAccSupp{method=escape,ActualText={<<-}}\(\twoheadrightarrow\)\EndAccSupp{}}}
\lstset{add to literate={<-}{{{\!\LeftArrow\!}}}2
   {<<-}{{{\!\DoubleLeftArrow\!}}}2
   {->}{{{\!\RightArrow\!}}}2
   {->>}{{{\!\DoubleRightArrow\!}}}2
}

\addtobeamertemplate{footline}{\begin{tikzpicture}[overlay,remember picture]
   \node[below left=4.65mm,btdm@softtext] at (current page.north east)
      {\Large\textbf{\MakeUppercase{uulm-sp}}};
\end{tikzpicture}}{}

\begin{document}
\titleframe<2->

\section[Feature Extraction]{Feature Extraction}
\def\Package#1{\T{\link{https://www.rdocumentation.org/packages/stats/#1}{#1}}}
\def\rotHead#1{\llap{\rotatebox{-52}{{\rlap{\tiny#1}}}}}
\newcommand<>\HItem[1]{\def\Mixin{30!white}\only#2{\def\Mixin{}}\begin{colormixin}{\Mixin}\item#1\end{colormixin}}
\colorlet{@back}{gray!90!white}
\def\Perc#1#2{\qty[round-precision=2,round-mode=places]{#1}\percent\llap{\raisebox{-5pt}{\scalebox{.6}{\small\color{@back}\num[group-digits,group-minimum-digits=4,group-separator={\,}]{#2}}}}}
\def\CurrentSum{1}
\newcommand\Print[2][\CurrentSum]{\FPdiv\Out{#2}{#1}\FPmul\Out{\Out}{100}\Perc{\Out}{#2}}
\begin{frame}[c]{\insertsection}
\def\arraystretch{1.15}\vspace*{-\medskipamount}%
\begin{columns}[onlytextwidth,c]
\column{.3\linewidth}
\begin{uncoverenv}<2->
   \begin{itemize}
      \itemsep6pt
      \HItem<handout:1|2,3>{Packages}
      \HItem<handout:2|2,4>{Roxygen2}
      \HItem<handout:3|2,5>{Function Definitions}
      \HItem<handout:4|2,6>{Function Calls}
      \HItem<handout:5|2,7>{Values}
      \HItem<handout:6|2,8>{Assignments}
      \HItem<handout:7|2,9>{Loops}
      \HItem<handout:8|2,10>{Controlflow}
      \HItem<handout:9|2,11>{Data-Access}
      \HItem<handout:10|2,12>{Meta}
   \end{itemize}
\end{uncoverenv}
\column{.69\linewidth}
\footnotesize\centering
\begin{onlyenv}<3|handout:1>
\def\P{23701}\def\U{20853}% Total Imports
% \def\P{25691}\def\ % based on files
\begin{tabular}{rl@{\hskip4em}rl}
   \multicolumn{4}{c}{\textbf{Top imported packages}} \\
   \multicolumn{2}{c}{\textit{Package Code}} & \multicolumn{2}{c}{\textit{User Code}} \smallskip\\
   \Print[\P]{1112} & \Package{stats}  & \Print[\U]{1212} & \Package{ggplot2}   \\
   \Print[\P]{934} & \Package{utils}   & \Print[\U]{1179} & \Package{dplyr}     \\
   \Print[\P]{848} & \Package{dplyr}   & \Print[\U]{772} & \Package{tidyverse} \\
   \Print[\P]{812} & \Package{rlang}   & \Print[\U]{623} & \Package{lme4}      \\
   \Print[\P]{802} & \Package{testhat} & \Print[\U]{455} & \Package{car}       \\
   \Print[\P]{674} & \Package{cli}     & \Print[\U]{408} & \Package{plyr}      \\
   \Print[\P]{661} & \Package{withr}   & \Print[\U]{319} & \Package{tidyr}     \\
\end{tabular}\bigskip\\
% \begin{tabular}{lrr}
%    \multicolumn{3}{c}{\textbf{Ways of Import}} \\
%    & \textit{Package Code} & \textit{User Code} \smallskip\\
% \T{::} & 15\,788 & 1\,876  \\
% \T{:::} & 637 & 42 \\
% \T{library} & 5\,224 & 17\,349 \\
% \T{require} & 911 & 1\,514 \\
% \T{requireNamespace} & 887 & 24 \\
% \textit{by variable} & 165  & 6 \\
% \T{loadNamespace} & 66 & 0 \\
% \T{*apply} & 21 & 42 \\
% \T{attachNamespace} & 2 & 0 \\
% \end{tabular}
\begin{tabular}{l *9{@{\hskip3pt}r}}
   \multicolumn{10}{c}{\textbf{Ways of import}} \smallskip\\
   \textit{Package} & \Print[\P]{15788} & \Print[\P]{637} & \Print[\P]{5224} & \Print[\P]{911} & \Print[\P]{887} & \Print[\P]{165} & \Print[\P]{66} & \Print[\P]{21} & \Print[\P]{2}\\
   \textit{User} & \Print[\U]{1876} & \Print[\U]{42} & \Print[\U]{17349} & \Print[\U]{1514} & \Print[\U]{24} & \Print[\U]{6} & \Print[\U]{0} & \Print[\U]{42} & \Print[\U]{0}\\
   & \rotHead{::} & \rotHead{:::} & \rotHead{library} & \rotHead{require} & \rotHead{req.Ns.} & \rotHead{by variable} & \rotHead{load Ns.} & \rotHead{*apply} & \rotHead{attach Ns.}
\end{tabular}\qquad~~
\end{onlyenv}
\begin{onlyenv}<4|handout:2>
\begin{tabular}{lrr}
   \multicolumn{3}{c}{\textbf{Comments}} \\
   & \textit{Package Code} & \textit{User Code} \smallskip\\
   Roxygen/Total &492\,510/884\,950 & 1\,486/311\,344 \smallskip\\
   \multicolumn{3}{c}{\textbf{Used Imports}} \\
   % & \textit{Package Code} & \textit{User Code} \smallskip\\
   \T{@import} & 305 & 8  \\
   \T{@importFrom} & 5\,587& 6 \\
   \T{@importClassesFrom} & 6 & 0\\
   \T{@importMethodsFrom} & 8 & 0\\
   \T{@useDynLib} & 605 & 0\smallskip\\
   \multicolumn{3}{c}{\textbf{Used Exports}} \\
   % & \textit{Package Code} & \textit{User Code} \smallskip\\
   \T{@export} & 33\,186 & 13  \\
   \T{@exportClass} & 27& 0 \\
   \T{@exportMethod} & 81& 0\\
   \T{@exportS3Method} & 14& 0\\
   \T{@exportPattern} & 1& 0\\
   % TODO: evalNamespace | rawNamespace
\end{tabular}
\end{onlyenv}
\begin{onlyenv}<5|handout:3>
\begin{tabular}{lrr}
% \multicolumn{3}{c}{\textbf{Defined Functions}} \\
& \textit{Package Code} & \textit{User Code} \smallskip\\
\T{Total} & 131\,953 & 12\,122  \\
\T{Lambdas} & 22 & 0 \\
\T{Assigned} & 96\,161& 6\,821\\
\T{Direct-Call} & 91& 0\\
\T{Nested} & 25\,521& 2\,240\\
\T{Recursive} & 1\,058& 10\\
\end{tabular}
\end{onlyenv}
% Function calls
\begin{onlyenv}<6|handout:4>
\begin{tabular}{rl@{\hskip4em}rl}
% \multicolumn{3}{c}{\textbf{Called Functions}} \\
\multicolumn{2}{c}{\textit{Package Code}} & \multicolumn{2}{c}{\textit{User Code}} \smallskip\\
1\,956\,887 & \textit{all} & 980\,964 & \textit{all} \\
133\,650 & \T{c}                & 110,020 & \T{c}   \\
 57\,398 & \T{list}             &  19,849 & \T{length}     \\
 56\,842 & \T{length}           &  18,975 & \T{library} \\
 50\,534  & \T{expect\_equal}   &  18,797 & \T{summary}      \\
 42\,279  & \T{is.null}         &  18,009 & \T{aes}       \\
 37\,238  & \T{test\_that}      &  16,303 & \T{list}      \\
 33\,688  & \T{return}          &  13,162 & \T{element\_text}     \\
\end{tabular}
\end{onlyenv}
\begin{onlyenv}<7|handout:5>
\begin{tabular}{lrr}
% \multicolumn{3}{c}{\textbf{Defined Functions}} \\
& \textit{Package Code} & \textit{User Code} \smallskip\\
{Numbers} & 1\,094\,489 & 844\,344  \\
\quad{Imaginary} & \Print[1094489]{1436} & \Print[844344]{37} \\
\quad{Integers} & \Print[1094489]{106843} & \Print[844344]{346} \\
\quad{FloatHex} & \Print[1094489]{12} & \Print[844344]{0} \\
{Logical} & 164\,249 & 64\,421\\
{Special} & 94\,590 & 20\,586\\ % NULL | NA | ...
{Strings} & 852\,103 & 488\,999\\
\end{tabular}\\
% TODO: Percentages?
\end{onlyenv}
\begin{onlyenv}<8|handout:6>
\def\P{772263}\def\U{396999}%
\begin{tabular}{lrr}
% \multicolumn{3}{c}{\textbf{Defined Functions}} \\
& \textit{Package Code} & \textit{User Code} \smallskip\\
{Operator}   & 772\,263 & 396\,999  \\
\quad{\LeftArrow}        & \Print[\P]{730160} & \Print[\U]{323286} \\
\quad{\DoubleLeftArrow}  & \Print[\P]{4211} & \Print[\U]{431} \\
\quad{=}                 & \Print[\P]{37048} & \Print[\U]{71769} \\
\quad{\RightArrow}       & \Print[\P]{145} & \Print[\U]{1004} \\ % grep -o "\"->\"" assignmentOperator.txt | wc -l
\quad{\DoubleRightArrow} & \Print[\P]{1} & \Print[\U]{0} \\
\quad{\textit{Others}}            & \Print[\P]{698} & \Print[\U]{509} \\
{Special} & 82\,800 & 86\,171\\ % with bracket
{Nested} & 6\,479 & 1\,680\\ % Directly
\end{tabular}
\end{onlyenv}
\begin{onlyenv}<9|handout:7>
\begin{tabular}{lrr}
% \multicolumn{3}{c}{\textbf{Defined Functions}} \\
& \textit{Package Code} & \textit{User Code} \smallskip\\
\T{for}    & 19\,298 & 13\,679 \\
\T{while}  & 1\,357 & 389 \\
\T{repeat} & 233 & 50 \\
\T{break} & 1\,112 & 205 \\
\T{next} & 932 & 282 \\
\end{tabular}
\end{onlyenv}
\begin{onlyenv}<10|handout:8>
\begin{tabular}{lrr}
% \multicolumn{3}{c}{\textbf{Defined Functions}} \\
& \textit{Package Code} & \textit{User Code} \smallskip\\
{if-then-else}    & 216\,722 & 13\,773 \\
\quad{nested} & 152\,022 & 9\,711 \\
\quad{constant} & 750 & 69 \\
\quad{variable} & 24\,844 & 723\\
switch-case & 3\,421 & 46 \\
\end{tabular}
\end{onlyenv}
\begin{onlyenv}<11|handout:9>
\begin{tabular}{lrr}
% \multicolumn{3}{c}{\textbf{Defined Functions}} \\
& \textit{Package Code} & \textit{User Code} \smallskip\\
\T{[}           & 186\,893 & 188\,888\\
\quad{empty}   &  1\,257 & 204\\
\quad{constant} & 43\,684 & 39\,587\\
\quad{variable} & 63\,805 & 42\,970\\
% \quad{comma} &  & \\
\T{[[}           & 62\,904 & 17\,248\\
\quad{empty}   & 10 & 0 \\
\quad{constant} & 30\,508 & 8\,601 \\
\quad{variable} & 29\,625  & 8\,011 \\
chained & 87\,134 & 90\,362\\
by name & 271\,577 & 283\,081\\% or nested
by slot & 19\,980 & 3\,198\\
\end{tabular}
\end{onlyenv}
\begin{onlyenv}<12|handout:10>
\begin{tabular}{lrr}
& \textit{Package Code} & \textit{User Code} \smallskip\\
Count & 25\,691 & 4\,230 \\
{File Length}           &  & \\
\quad{min} & 1 & 1 \\
\quad{max} & 29\,211 & 7\,100\\
\quad{avg} & 145 & 348 \\
\quad{median} & 70 & 184 \\
{Line Length}           &  & \\
\quad{min} & 0 & 0 \\
\quad{max} & 116\,068 & 33\,341 \\
\quad{avg} & 32 & 44 \\
\quad{median} & 28 & 36 \\
\end{tabular}
\end{onlyenv}
\end{columns}
\end{frame}

\section[Quad Generation]{Quad Generation}
\begin{frame}[fragile]{\insertsection}
\begin{itemize}
   \item<2-> We can generate Quads from the normalized AST
   \item<3-> Limited to files that can be parsed
\end{itemize}
\bigskip
\begin{uncoverenv}<4->
\lstfs{5}\soldisablenumhl
\begin{minted}{void}
<https://uni-ulm.de/r-ast/filename/0> <https://uni-ulm.de/r-ast/type> "exprlist" <filename> .
<https://uni-ulm.de/r-ast/filename/0> <https://uni-ulm.de/r-ast/children-0> <https://uni-ulm.de/r-ast/filename/1> <filename> .
<https://uni-ulm.de/r-ast/filename/1> <https://uni-ulm.de/r-ast/location> <https://uni-ulm.de/r-ast/filename/2> <filename> .
<https://uni-ulm.de/r-ast/filename/2> <https://uni-ulm.de/r-ast/start> <https://uni-ulm.de/r-ast/filename/3> <filename> .
<https://uni-ulm.de/r-ast/filename/3> <https://uni-ulm.de/r-ast/line> "1"^^<http://www.w3.org/2001/XMLSchema#integer> <filename> .
<https://uni-ulm.de/r-ast/filename/3> <https://uni-ulm.de/r-ast/column> "1"^^<http://www.w3.org/2001/XMLSchema#integer> <filename> .
<https://uni-ulm.de/r-ast/filename/2> <https://uni-ulm.de/r-ast/end> <https://uni-ulm.de/r-ast/filename/4> <filename> .
<https://uni-ulm.de/r-ast/filename/4> <https://uni-ulm.de/r-ast/line> "1"^^<http://www.w3.org/2001/XMLSchema#integer> <filename> .
<https://uni-ulm.de/r-ast/filename/4> <https://uni-ulm.de/r-ast/column> "1"^^<http://www.w3.org/2001/XMLSchema#integer> <filename> .
<https://uni-ulm.de/r-ast/filename/1> <https://uni-ulm.de/r-ast/lexeme> "1" <filename> .
<https://uni-ulm.de/r-ast/filename/1> <https://uni-ulm.de/r-ast/type> "NUM_CONST" <filename> .
\end{minted}
\end{uncoverenv}\bigskip
\begin{itemize}
   \item<5-> Question: Which domain should we use?
\end{itemize}
\end{frame}

\section[Ontology]{Ontology}
\begin{frame}{\insertsection}
\begin{uncoverenv}<2->
\vspace*{-1.5\baselineskip}\tiny\begin{forest}
   for tree={font=\ttfamily,grow=east,anchor=west,child anchor=west,l sep=2cm}
   [Base
      [RUnaryOp,grow=east,anchor=west,child anchor=west[RLogicalUnaryOp][RArithmeticUnaryOp]]
      [RExpressionList]
      [RFunctionCall]
      [RRepeatLoop]
      [RIfThenElse]
      [RWhileLoop]
      [RForLoop]
      [RFunctionDefinition]
      [RArgument]
      [RComment]
      [RLogical]
      [RSymbol]
      [RBreak]
      [RNumber]
      [RString]
      [RBinaryOp,grow=east,anchor=west,child anchor=west[RAssignmentOp][RComparisonBinaryOp][RArithmeticBinaryOp][RLogicalBinaryOp]]
   ]
\end{forest}
\end{uncoverenv}
\begin{tikzpicture}[overlay,remember picture]
   \onslide<3->{\node[left=5mm,yshift=-5mm,font=\scriptsize\sffamily,color=gray] at(current page.east) {\link{https://github.com/Code-Inspect/ontology/issues/1}{https://github.com/Code-Inspect/ontology/issues/1}};}
\end{tikzpicture}
\end{frame}

\end{document}