\documentclass{sopra-base}

\usepackage{sopra-documentation}
\usepackage{sopra-ntts}

\title{Das 'sopra-ntts'-Paket}
\subtitle[Dokumentation für das 'sopra-ntts'-Paket]{Dokumentation für das 'sopra-ntts'-Paket | Version \thesonversion}

\duedate{2019-12-2}

\keywords{Dokumentation,sopra-ntts,sopra,uni ulm,uulm,Paket}
\authors{Florian Sihler (florian.sihler@uni-ulm.de)}

\group{Die Affenbande}
\begin{document}

    \maketitle

\section{Allgemeines}
\subsection{Warum, wieso, weshalb?}
    Dieses \LaTeXe-Paket lagert die Futura-ähnliche Ti\textit{k}Z Font aus dem Sopra des
    Wintersemester 2019 und Sommersemester 2020 aus.
    Diese Dokumentation wurde zusammen mit der
    \T{sopra-base.cls}, sowie dem Paket \T{sopra-documentation.sty} kreiert.\par
    Zum Visualisieren der einzelnen Code-Ausschnitte wird das
    \T{sopra-listings}-Paket verwendet.
    Das zugehörige Paket sollte ebenfalls in dieses Dokument eingebettet sein: \scalebox{.65}{\attachfile[subject={sopra-ntts.sty}]{sopra-ntts.sty}}.
\subsection{Abhängigkeiten}
    Dieses Paket bindet die folgenden Paketen mit ein:
    \begin{multicols}{3}
        \begin{itemize}
            \def\pkgparse#1:#2\@nil{%
                \T{#1}\ifx!#2!\else\textsuperscript{(#2)}\fi%
            }
            \foreach \pkg in {tikz:} {
                \item \expandafter\pkgparse\pkg\@nil
            }
        \end{itemize}
    \end{multicols}
    All diese Pakete sollten Teil der gängigen \LaTeX-Distribution sein.

\subsection{Die Installation}
    Das Paket wird nicht als \T{.dtx} ausgeliefert, weswegen sich die
    folgenden Möglichkeiten ergeben:
    \begin{itemize}
        \item Das Paket kann in dasselbe Verzeichnis wie das Dokument
                gesetzt werden. In diesem Fall lautet die Einbindungsanweisung:
\begin{plainlatex}
\usepackage{sopra-ntts}
\end{plainlatex}
        \item Das Paket kann in ein Unterverzeichnis/in ein mit
                dem Dokument ausgeliefertes Verzeichnis gelegt werden. In
                diesem Fall erfolgt die Angabe durch den (relativen-) Pfad:
\begin{plainlatex}
\usepackage{./Mein/Pfad/zu/sopra-ntts}
\end{plainlatex}
        \item Man kann das Paket (mittels eines Symlinks oder ähnlichem)
              in einen eigenen \emph{texmf}-Baum ablegen.
              So kann zum Beispiel auf Linux unter der Verwendung von texlive
              das Paket hier abgelegt werden: \bvoid{\~/texmf/tex/latex/}.
              Das Verzeichnis kann erstellt und anschließend mittels
              \bbash{texhash \~/texmf} aktualisiert werden. Nun kann
              das Paket wie jede andere installierte Paket verwendet werden:
\begin{plainlatex}
\usepackage{sopra-ntts}
\end{plainlatex}
    \end{itemize}

\subsection{Weitere Besonderheiten}
In Version \thesonversion{} (\cmdref{thesonversion}) gibt es keine weiteren
Besonderheiten.

\section{Befehle- und Umgebungen}

Es gilt zu beachten, dass das Präfix \T{env@} nur auf die Natur einer Umgebung hinweist und nicht zum eigentlichen Bezeichner zuzuordnen ist!\par{}
Weiter gilt: Damit alle Titel und Längen richtig aufgelöst werden können, muss das Dokument
in der Regel zwei mal kompiliert werden um eine korrekte Anzeige zu erzeugen.

\begin{command}{sonNtts}{\optArg{scale=0.06}\manArg{color}}
    Muss in eine \say{tikzpicture}-Umgebung gesetzt werden und schreibt den Schriftzug (hier mit schwarzer Farbe):
    \begin{center}
        \begin{tikzpicture}
            \sonNtts{black}
        \end{tikzpicture}
    \end{center}
    Der Befehl wurde nie grundlegend für die allgemeine Nutzung konzipiert. Die Verwendung der Koordinaten findet sich in der Definition der Titelseiten der Dokumentklassen.
\end{command}

\subsection{Allgemeine Befehle}
\begin{command}{thesonversion}{}
    Liefert die aktuelle Version des Pakets. So ergibt: \cmd{thesonversion}: \thesonversion\\
    \notetext{Hinweis: über \blatex{\\value\{sonversion\}} lässt sich
    die Version als $4$-stellige Nummer erhalten: \arabic{sonversion}.}
\end{command}

\end{document}