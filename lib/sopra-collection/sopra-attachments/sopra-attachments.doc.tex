\documentclass{sopra-base}

\usepackage{sopra-documentation}
\usepackage{sopra-attachments}

\title{Das 'sopra-attachments'-Paket}
\subtitle[Dokumentation für das 'sopra-attachments'-Paket]{Dokumentation für das 'sopra-attachments'-Paket | Version \thesoaversion}
\duedate{2020-01-5}

\keywords{Dokumentation,sopra-attachments,sopra,uni ulm,uulm,Paket}
\authors{Florian Sihler (florian.sihler@uni-ulm.de)}

\group{Die Affenbande}
\begin{document}
    \maketitle

\section{Allgemeines}
\subsection{Warum, wieso, weshalb?}
    Dieses \LaTeXe-Paket wurde im Rahmen des Sopras im
    Wintersemester 2019 und Sommersemester 2020 verfasst und dient als
    kleiner Wrapper zum Einbinden von Dokumenten
    des \imptext{Teams 20}. Diese Dokumentation wurde zusammen mit der
    \T{sopra-base.cls} sowie dem Paket \T{sopra-documentation.sty} kreiert.\par
    Zum Visualisieren der einzelnen Code-Ausschnitte wird das
    \T{sopra-listings}-Paket verwendet.
    Das zugehörige Paket sollte ebenfalls in dieses Dokument eingebettet sein: \attachTexText{sopra-attachments.sty}{sopra-attachments.sty}.
\subsection{Abhängigkeiten}
    Dieses Paket bindet die folgenden Paketen mit ein:
    \begin{multicols}{3}
        \begin{itemize}
            \def\pkgparse#1:#2\@nil{%
                \T{#1}\ifx!#2!\else\textsuperscript{(#2)}\fi%
            }
            \foreach \pkg in {xcolor:,attachfile2:,graphicx:{\jmark[\T{attsymbols}]{mrk:attsymbols}, \jmark[\T{usefa}]{mrk:usefa}},fontawesome:{\jmark[\T{usefa}]{mrk:usefa}}} {
                \item \expandafter\pkgparse\pkg\@nil
            }
        \end{itemize}
    \end{multicols}
    All diese Pakete sollten Teil der gängigen \LaTeX-Distribution sein.

\subsection{Die Installation}
    Das Paket wird nicht als \T{.dtx} ausgeliefert, weswegen sich die
    folgenden Möglichkeiten ergeben:
    \begin{itemize}
        \item Das Paket kann in dasselbe Verzeichnis wie das Dokument
                gesetzt werden. In diesem Fall lautet die Einbindungsanweisung:
\begin{plainlatex}
\usepackage{sopra-attachments}
\end{plainlatex}
        \item Das Paket kann in ein Unterverzeichnis/in ein mit
                dem Dokument ausgeliefertes Verzeichnis gelegt werden. In
                diesem Fall erfolgt die Angabe durch den (relativen-) Pfad:
\begin{plainlatex}
\usepackage{./Mein/Pfad/zu/sopra-attachments}
\end{plainlatex}
        \item Man kann das Paket (mittels eines Symlinks oder ähnlichem)
              in einen eigenen \emph{texmf}-Baum ablegen.
              So kann zum Beispiel auf Linux unter der Verwendung von texlive
              das Paket hier abgelegt werden: \bvoid{\~/texmf/tex/latex/}.
              Das Verzeichnis kann erstellt und anschließend mittels
              \bbash{texhash \~/texmf} aktualisiert werden. Nun kann
              das Paket wie jede andere installierte Paket verwendet werden:
\begin{plainlatex}
\usepackage{sopra-attachments}
\end{plainlatex}
    \end{itemize}

\subsection{Weitere Besonderheiten}
In Version \thesoaversion{} (\cmdref{thesoaversion}) gilt: es wird keine Standardfarbe mehr gesetzt um es einefacher zu machen diese selbst zu konfigurieren. Weiter wurde \argref{usefa}{nousefa} hinzugefügt.

\subsection{Akzeptierte Parameter}
    Das Paket akzeptiert, so wie die meisten, Argumente.
    Bei Argumenten mit einer \say{Counter}-Option wird das jeweils standardmäßig aktive zuerst und das andere in Klammern
    geschrieben. So wird implizit:
\begin{plainlatex}
    \usepackage[attsymbols,nodocheck,usefa]{sopra-attachments}
\end{plainlatex}
    aufgerufen. Während wir mit:
\begin{plainlatex}
    \usepackage[docheck]{sopra-attachments}
\end{plainlatex}
    kein Fehler geworfen wird wenn die Datei nicht existiert!

    \begin{argument}{attsymbols}{noattsymbols}
        Mit \T{noattsymbols} wird auch dann, wenn eigentlich ein Symbol verwendet werden
        soll auf Text zurückgegriffen.
    \end{argument}

    \begin{argument}{nodocheck}{docheck}
        Mit \T{docheck} wird geprüft, ob es die Datei auch gibt. Sie wird nur im Existenzfall
        angehängt, andernfalls wird \emph{kein} Fehler geworfen. Dies kann zum Beispiel dann
        nützlich sein, wenn man die einzubindende Grafik mit Ti\textit{k}Z-\T{external} erst
        später im Dokument generiert.
    \end{argument}

    \begin{argument}{usefa}{nousefa}
        Mit \T{usefa} werden die Pakete \emph{fontawesome} sowie \emph{graphicx} geladen. Vor den angehängten Dateien wird bei den Text-Kommandos wie \cmdref{attachDocumentText} eine Papierklammer vorangestellt um das Anhängen deutlich zu machen. Mit \T{nousefa} erhält man das Verhalten vor Version \T{v1.1.00} zurück.
    \end{argument}

\subsection{Weitere Hinweise}

    Mittels \cmd{attachfilesetup} des Pakets \T{attachfile2} lassen sich weitere
    Einstellungen vornehmen. Wird nicht \T{sopra-base} verwendet, muss man den Author
    der Dokumente durch \cmd{author} setzen.

\section{Befehle- und Umgebungen}

\subsection{Allgemeine Befehle}

\begin{command}{thesoaversion}{}
    Liefert die aktuelle Version des Pakets. So ergibt: \cmd{thesoaversion}: \thesoaversion\\
    \notetext{Hinweis: über \blatex{\\value\{soaversion\}} lässt sich
    die Version als $4$-stellige Nummer erhalten: \arabic{soaversion}.}
\end{command}

\begin{command}{attachPdf}{\optArg{Suchpfad}\manArg{Datei}}
    Versucht \T{Datei} einzubinden. Befindet sie sich nicht im gleichen Ordner, so kann
    \T{Suchpfad} als Präfix angegeben werden. \notetext{Ein \say{slash} am Ende wird
    erwartet!}
\end{command}

\begin{command}{attachPdfText}{\optArg{Suchpfad}\manArg{Datei}\manArg{Text}}
    Analog zu \cmdref{attachPdf}, allerdings wird anstelle des Symbols (sofern aktiviert),
    der \T{Text} angezeigt.
\end{command}

\begin{command}{attachDocument}{\optArg{Suchpfad}\manArg{Datei}}
    Analog zu \cmdref{attachPdf}, allerdings wird der Dokumenttyp auf \T{text/plain} gesetzt. \notetext{Es existiert das Alias \cmd{attachTex}.}
\end{command}

\begin{command}{attachDocumentText}{\optArg{Suchpfad}\manArg{Datei}\manArg{Text}}
    Analog zu \cmdref{attachPdfText}, allerdings wird der Dokumenttyp auf \T{text/plain} gesetzt. \notetext{Es existiert das Alias \cmd{attachTexText}.}
\end{command}

\end{document}